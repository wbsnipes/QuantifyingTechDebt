The idea presented in this paper provides a framework for continuously estimating the interest payments on items representing technical debt.  We define a low level view of interest as the time required for developers to comprehend a class as they are working on code within the class or related to the class.   

The result of this is we assign interest payments to all code meeting the criteria that \TD cannot be completely eliminated from code.  This framework also views \TD as dependent upon the maintenance and evolution activity in the code base, meeting the criteria that \TD could result from a context shift requiring stable code to be revised~\cite{Ozkaya2012Technical}. By continuously assessing the interest payments on \TD the framework help teams prioritize debt removal efforts in real-time for code with the highest cost.  We seek to have a balance between pure cost to implement changes with cost to comprehend by structuring measurements around comprehension sessions for each class.

The comprehension effort relates to \TD through occurrence of code smells in one dimension.  For example, consider the code smell of Feature Envy where a method makes too many calls to other classes to obtain data or functionality.  By calculating from $Blaze$ data the number and time spent visiting other classes within a session, we can estimate the effort required to understand dependencies by the developer.  The total count of visits to other classes relates to the Feature Envy smell or Inappropriate Intimacy.  The number of classes visited (unique classes) in each session could  relate to the Shotgun Surgery smell particularly when multiple classes are edited in a session.  The time spent visiting classes and time per class visited may relate to the Long Class smell \cite{Fowler_etal:1999}.  

In order to confirm the above idea, we collect source code metrics from the code being viewed and edited by the developers.  Using the work of Nugroho et al. who identify potential code maintainability issues from source metrics~\cite{Nugroho2011Empirical}, we consider how the observed class visit events in our comprehension data relate to static code metrics provided by the Understand tool from Scientific Tool Works\footnote{www.scitools.com}.   
