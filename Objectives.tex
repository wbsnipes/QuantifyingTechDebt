\Fix{NK: I think that this section should be titled ``Overview'' and should start with a paragraph that defines what a ``session'' (or ``comprehension session'') is and provides a high-level description of how code smells and comprehension cost fit in (and relate to principal and interest).}

In this paper we propose a framework for continuously estimating the interest payments on items representing technical debt. \Fix{Will:this is duplicate info} We define a low-level view of interest as the time required for developers to comprehend a class as they are working on code inside of or related to the class. As a result, we assign interest payments to all code meeting the criteria that \TD cannot be completely eliminated from code. \Fix{NK: Need to clarify the previous sentence. It is not entirely clear to me what the criteria is and how it relates to the low-level view.}

Within the proposed framework, we consider \TD to be dependent on the maintenance and evolution activity in the code base, meeting the criteria that \TD could result from a context shift requiring stable code to be revised~\cite{Ozkaya_etal:2012}. \Fix{NK: Again, the ``meeting the criteria'' bit is not clear to me.} By continuously assessing the interest payments on \TD, the framework enables teams to prioritize debt removal efforts in real-time.  Further, by measuring code comprehension on a per-class basis, we permit fine-grained analysis of the cost to implement a change versus the cost to comprehend the requisite classes. \Fix{NK: I want to make the previous sentence more specific but am unsure of the original intent. Does the ``cost to implement a change'' refer to the principal or interest?}

The comprehension effort relates to \TD through occurrence of code smells in one dimension.  For example, consider the code smell of Feature Envy where a method makes too many calls to other classes to obtain data or functionality.  By calculating from $Blaze$ data the number \Fix{NK: The number of what?} and time spent visiting other classes within a session \Fix{The concept of a session needs to be introduced/defined in the previous paragraph}, we can estimate the developer effort required to understand dependencies.  The total count of visits to other classes relates to the Feature Envy smell or Inappropriate Intimacy. Further, the number of (unique) classes visited in each session could  relate to the Shotgun Surgery smell, particularly when multiple classes are edited in a session. The time spent visiting classes and time per class visited may relate to the Long Class smell~\cite{Fowler_etal:1999}.  

\Fix{Will: wording this next phrase}
In order to confirm the above idea, we collect source code metrics from the code being viewed and edited by the developers.  Similar to Nugroho et al.~\cite{Nugroho_etal:2011}, in which potential code maintainability issues are identified using static code metrics, we consider how the observed class-visit events in our comprehension data relate to static code metrics provided by $Understand$\footnote{\url{http://www.scitools.com}}.