\begin{center}
\begin{table*}[t]
	\centering
	\caption{Class Data for Each Developer}
	\begin{tabular}{|l|l|l|l|l|l|l|}
	\hline

Developer Name & Class Name & No Of Blocks &Count No of Class Edits & Count No of Other Class Accesses & Time Spent in Class & Time Spent in Other Classes\\
\hline\hline
Developer X & A & 74 & 294 & 78 & 39hours 34mins 23 secs & 16hours 39mins 36secs\\
\hline
Developer X & B & 32 & 88 & 52 & 2hours 54mins 20 secs & 8hours 54mins 18secs\\
\hline
Developer X & C & 28 & 77 & 52 & 4hours 28mins 25 secs & 9hours 15mins 41secs\\
\hline
Developer Y & P & 10 & 45 & 24 & 2hours 19mins 13 secs & 1hour 43mins 12secs\\
\hline
Developer Y & Q & 7 & 12 & 20 & 24 mins 08 secs & 2hours 36mins 32secs\\
\hline
Developer Y & R & 5 & 15 & 16 & 53 mins 51 secs & 40mins 18secs\\
\hline

	\end{tabular}
	\label{fig:AnalysisData}
\end{table*}
\end{center}

As previously mentioned we first define sessions to investigate each class that a developer visits. We define session as a moving window time where developer is investigating a certain class. When we define a session of X hours for a class Y we mean to say that starting from the first time the developer visits class Y while navigating we find the last time he visits the same class in X hours. We refer to this unit as a  "block" for the class that we want to study. For each block of class Y that we obtain we study the number of unique classes the developer has visited while in the block as well as the number of times the developer has visited the class Y iteslf within the block. The classes central to a task will have a high count of the number of times the developer visits the class in a particular block and the number of such blocks will also be high. Also the time spent editing such a class will also be higher than time spent in most of the other classes within the block. The time factor gives us an estimation of how well the developer comprehends the class and other classes referenced to in the block. We studied such blocks for a moving window of 4 hours, 8 hours, 12 hours and 16 hours. 

The data collected by the Blaze logs has all the navigation activities done by the developer in the IDE. It also has appropriate time outs for periods of inactivity in the IDE. We first filtered out all the navigation activities related to classes as well all time outs and IDE exits. Next we calculated the time spent by the developer in the file. We obtained a list of such files and the time spent by the user in each file. We observed there were instances when the developer accessed a file for less than a second and then switched to another file. We assumed that no useful comprehension can be done by the user in less than a second and attributed such file accesses to random clicks and removed all such entries. We then calculated various parameters that would help in identifying the central class as well the neccesary classes for comprehending the central class. For each sliding window size we calculated the following parameters:
\begin{itemize}
	\item[] Number of blocks formed by each class: No Of Blocks.
	\item[] Number of times the class itself has been accessed inside the block : Count No of Class Edits.
	\item[] Number of unique files visited in each block: Count No of Other Class Accesses.
	\item[] Time spent in the class that forms the block: Time Spent in Class.
	\item[] Time spent in all other files in the block: Time Spent in Other Classes . 
\end{itemize}
We observed that there was a small change in the number of blocks formed by each class between the 4 hour moving window and the 8 hour moving window. The fact that there is not a huge downslide from the 4 hour window to the 8 hour window means that developers do not very often work in 8 hour windows but rather more often do so in 4 hour windows. The 8 hour moving window formed the same number of blocks as that of the 12 hour and 16 hour window. Since the difference was not much between the 4 hour and 8 hour sliding window we decided to use the 4 hour sliding window for further analysis.

Table~\ref{fig:AnalysisData} shows data for each developer that we investigated. We present 3 classes for each of the developers. Each column represents the 5 parameters we defined to be useful in quantifying the \TD. As you can see for developer X we observe a very high count of the number of blocks as well as high counts in number of class edits and number of other class accesses. This suggests that the developer referenced Class A over a long period of time and very frequently. The developer also frequently referred to other classes while working on Class A. The data shows that the developer spent more than 39 hours working on Class A and more than 16 hours referencing other classes. The 16 hours the developer spent on other files can be interpreted as a cost of comprehending class A. Thus it can be veiwed as  \TD. Similarly when we see Class B we see that developer X spent nearly 3 hours on Class B but spent nearly 9 hours referencing other classes. This indicates that the developer spent 3 times the time actually spent on the central class in other classes. Only taking time as a factor cannot lead to such conclusions and we need to make sure that the class that we consider as the block is central to the task at hand. In this case the developer has 32 such blocks of Class B where he has also referenced Class B 88 times in these blocks. We can safely say that the developer keeps coming back to class B and thus it must be central to the task. Similar patterns can be seen in developer Y where each parameter is indicative of the fact that the considered class is indeed the central class and that the time spent navigating other classes has a significant impact in calculating \TD.
